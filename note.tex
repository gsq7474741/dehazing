\documentclass[UTF8]{ctexart}

\usepackage{amsmath}
\usepackage{hyperref}
\usepackage{subfigure}
\usepackage[graphicx]{realboxes}


\title{Dehazing Learning Note}
\author{gsq}
\date{\today}

\begin{document}

\maketitle
\tableofcontents

\section{LaTeX using tips}

Some tips when writing this article.

\subsection{Document Organization}
在文档类 article/ctexart 中,定义了五个控制序列来调整行文组织结构。他们分别是

\begin{itemize}
    \item section
    \item subsection
    \item subsubsection
    \item paragraph
    \item subparagraph
\end{itemize}

在report/ctexrep中,还有chapter;在文档类book/ctexbook中,还定义了part。


\section{Dataset}

\subsection{Day}



\subsection{Night}

\subsubsection{NHC~\cite{3r}}

550 clear images were selected from Cityscapes to synthesize nighttime hazy images using 3R.
We synthesized 5 images for each of them by changing the light positions and colors, 
resulting in a total of 2,750 images, called “Nighttime Hazy Cityscapes” (NHC).
We also altered the haze density by setting $\beta t$ to 0.005, 0.01, and 0.02,
resulting in different datasets denoted NHC-L, NHC-M, and NHC-D, 
where “L”, “M”, and “D” represent light haze, medium haze, and dense haze. 
Further, we also modified the method by changing the constant yellow light color with our 
randomly sampled real-world light colors described in Section 3.1 and synthesized images 
on the Middlebury (70 images) and RESIDE (8,970 images) datasets.
Similar to NHC, we augmented the Middlebury dataset by 5 times, resulting in a total
of 350 images.
They are denoted NHM and NHR, respectively.
The statistics of these datasets are summarized in Table 1 in Section 6

\begin{table}[]
    \centering
    \begin{tabular}{lllll}
    \hline
    Dataset & Haze Density & Number & Synthetic & Tasks \\ \hline
    NHC-L   & Light        & 2750   & ✓         & ND+CR \\
    NHC-M   & Medium       & 2750   & ✓         & ND    \\
    NHC-D   & Dense        & 2750   & ✓         & ND    \\
    NHM     & All          & 350    & ✓         & ND    \\
    NHR     & All          & 8970   & ✓         & ND    \\
    NHRW    & All          & 150    & x         & ND    \\
    DCRW    & x            & 1500   & x         & CR    \\ \hline
    \end{tabular}
    \end{table}

\begin{itemize}
    \item NHC-M medium haze (2,750 images)
    \item NHC-L light haze
    \item NHC-D dense haze
    \item NHM Middlebury (70 images)
    \item NHR RESIDE (8,970 images)
\end{itemize}

\subsubsection{NightHaze \& YellowHaze~\cite{hdp}}
We also provide the datasets adopted in the work:
\begin{itemize}
    \item Haze-Free
    \item NightHaze-1
    \item NightHaze-2
    \item Haze-Free-Yellow
    \item YellowHaze-1
    \item YellowHaze-2
\end{itemize}

The initial size of images is not 128 x 128 but they are
downsampled to smaller resolution before training.
Dehazing effects may vary when a different dataset is trained.



\begin{thebibliography}{8}
    \bibitem{3r}
    Zhang, Jing, et al. "Nighttime dehazing with a synthetic benchmark." Proceedings of the 28th ACM international conference on multimedia. 2020.

    \bibitem{hdp}
    Liao, Yinghong, et al. "Hdp-net: Haze density prediction network for nighttime dehazing." Pacific Rim Conference on Multimedia. Springer, Cham, 2018.

\end{thebibliography}

\end{document}
